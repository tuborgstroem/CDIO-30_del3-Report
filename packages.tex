%%%%% PACKAGES %%%%%%
\usepackage[utf8]{inputenc}
\usepackage{lastpage}						% Giver mulighed for "side x af y"
\usepackage{url}							% Giver mulighed for links i fodnote
\usepackage{array}
\usepackage{calc}
\usepackage{mwe}	
\usepackage[nomessages]{fp}
\usepackage{booktabs} 						% Punktopstilling til mufs bachelor
\usepackage{appendix}
\usepackage{lmodern}
\usepackage{multicol}
\usepackage{stackengine}
\usepackage{tcolorbox}
\usepackage{graphicx}
\usepackage{subcaption}
\usepackage{pdfpages}
\usepackage{adjustbox}
\usepackage{xcolor}
\usepackage{smartdiagram}
\usepackage{url}
\usepackage{hyperref}
\usepackage{subcaption}
\usepackage{blindtext}
\usepackage[T1]{fontenc}
\usepackage{etoolbox}
\usepackage{amsmath}
\makeatletter
\patchcmd{\chapter}{\if@openright\cleardoublepage\else\clearpage\fi}{}{}{}
% \patchcmd{<cmd>}{<search>}{<replace>}{<success>}{<failure>}
% --- Patch \chapter
\patchcmd{\@makechapterhead}{1\p@}{\chapheadtopskip}{}{}% Space from top of page to CHAPTER X
\patchcmd{\@makechapterhead}{20\p@}{\chapheadsep}{}{}% Space between CHAPTER X and CHAPTER TITLE
\patchcmd{\@makechapterhead}{40\p@}{\chapheadbelowskip}{}{}% Space between CHAPTER TITLE and text
% --- Patch \chapter*
\patchcmd{\@makeschapterhead}{50\p@}{\chapheadtopskip}{}{}% Space from top of page to CHAPTER TITLE
\patchcmd{\@makeschapterhead}{40\p@}{\chapheadbelowskip}{}{}% SPace between CHAPTER TITLE and text
\makeatother
% Set new lengths
\newlength{\chapheadtopskip}\setlength{\chapheadtopskip}{20pt}
\newlength{\chapheadsep}\setlength{\chapheadsep}{40pt}
\newlength{\chapheadbelowskip}\setlength{\chapheadbelowskip}{15pt}
\makeatother
\usepackage{tasks}
\usepackage[ampersand]{easylist}


%%%%% TIKZ %%%%%
\usepackage{tikz}
\usetikzlibrary{circuits.ee.IEC,shapes,arrows,spy,positioning,snakes,plotmarks,backgrounds}
\usepackage[american voltages, american currents,siunitx,smartlabels]{circuitikz}
\usepackage{pgfplots}

%%%%% OVERSAETTELSE OG TEGNSAETNING %%%%%
%\usepackage[danish]{babel}					% Dokumentets sprog
\usepackage[T1]{fontenc}					% Output-indkodning af tegnsaet (T1)
\usepackage{ragged2e,anyfontsize}			% Justering af elementer
				
																					
%%%%% FIGURES AND TABLES %%%%%
\usepackage{pbox}							% Tvungne linjeskift i tabeller
\usepackage{tabularx}						% Tvungne linjeskift i tabeller
\usepackage{graphicx} 						% Haandtering af eksterne \nomenclature[W]{$T_{Melt}$}{Melt temperature [\si{\celsius}]}befbilleder (JPG, PNG, EPS, PDF)
\usepackage{eso-pic}						% Tilfoej billedekommandoer paa hver side
\usepackage{multirow}                		% Fletning af raekker og kolonner (\multicolumn og \multirow)
\usepackage{multicol}         	        	% Muliggoer output i spalter
\usepackage{rotating}						% Rotation af tekst med \begin{sideways}...\end{sideways}
\usepackage{colortbl} 						% Farver i tabeller (fx \columncolor og \rowcolor)
\usepackage{xcolor}							% Definer farver med \definecolor
\usepackage{flafter}						% Soerger for at floats ikke optraeder i teksten foer deres reference
\let\newfloat\relax 						% Justering mellem float-pakken og memoir
\usepackage{float}							% Muliggoer eksakt placering af floats, f.eks. \begin{figure}[H]
\graphicspath{{./figures/}}					% Saetter default grafik-sti
\usepackage{placeins}						% \FloatBarrier
\usepackage{wrapfig}
\usepackage{varwidth} 

%%%%% MATEMATIK MED MERE %%%%%
\usepackage{amsmath,amssymb,stmaryrd} 		% Avancerede matematik-udvidelser
\usepackage{mathtools}						% Andre matematik- og tegnudvidelser
\usepackage{textcomp}                 		% Symbol-udvidelser
\usepackage{rsphrase}						% Kemi-pakke til RS-saetninger,
\usepackage[version=3]{mhchem} 				% Kemi-pakke til flot og let notation af formler
\usepackage{siunitx}						% Flot og konsistent praesentation af tal og enheder med \si{enhed} og \SI{tal}{enhed}
%\sisetup{locale=DE}						% Opsaetning af \SI (DE for komma som decimalseparator) 
\sisetup{output-decimal-marker = {.}}		% Bruges ved engelske rapporter

%%%%% REFERENCES AND SOURCES %%%%%
\usepackage[danish]{varioref}				% Muliggoer bl.a. krydshenvisninger med sidetal (\vref)
\usepackage{xr}								% Referencer til eksternt dokument med \externaldocument{<NAVN>}
\usepackage{glossaries}						% Terminologi- eller symbolliste (se mere i Daleifs Latex-bog)

%%%%% MISC. %%%%%%
\usepackage{titlesec}
\usepackage{epigraph}
\usepackage{lipsum}							% Dummy text \lipsum[..]
\usepackage[shortlabels]{enumitem}			% Muliggoer enkelt konfiguration af lister
\usepackage{pdfpages}						% Goer det muligt at inkludere pdf-dokumenter med kommandoen \includepdf[pages={x-y}]{fil.pdf}	
\pretolerance=2500 							% Justering af afstand mellem ord (hoejt tal, mindre orddeling og mere luft mellem ord)
\usepackage[footnote,draft,english,silent,nomargin]{fixme}		


\pgfplotsset{compat=1.15}





% code in latex
\usepackage{listings}
\usepackage{color}

\definecolor{chaptercolor}{rgb}{0.36, 0.54, 0.66}
\definecolor{mygreen}{RGB}{28,172,0} % color values Red, Green, Blue
\definecolor{mylilas}{RGB}{170,55,241}

\definecolor{dkgreen}{rgb}{0,0.6,0}
\definecolor{gray}{rgb}{0.5,0.5,0.5}
\definecolor{mauve}{rgb}{0.58,0,0.82}

\lstset{frame=tb,
  language=python,
  aboveskip=3mm,
  belowskip=3mm,
  showstringspaces=false,
  columns=flexible,
  basicstyle={\small\ttfamily},
  numbers=none,
  numberstyle=\tiny\color{gray},
  keywordstyle=\color{blue},
  commentstyle=\color{dkgreen},
  stringstyle=\color{mauve},
  breaklines=true,
  breakatwhitespace=true,
  tabsize=3
}

%\usepackage[utf8]{inputenc}
\usepackage{listingsutf8}
\lstset{language=Matlab,%
    %basicstyle=\color{red},
    breaklines=true,
    extendedchars=true,
    morekeywords={matlab2tikz},
    keywordstyle=\color{blue},%
    morekeywords=[2]{1}, keywordstyle=[2]{\color{black}},
    identifierstyle=\color{black},%
    stringstyle=\color{mylilas},
    commentstyle=\color{mygreen},%
    showstringspaces=false,%without this there will be a symbol in the places where there is a space
    numbers=left,%
    numberstyle={\tiny \color{black}},% size of the numbers
    numbersep=9pt, % this defines how far the numbers are from the text
    emph=[1]{for,end,break},emphstyle=[1]\color{blue}, %some words to emphasise
    %emph=[2]{word1,word2}, emphstyle=[2]{style},    
}



% Nomenclature grouping and creation %
\usepackage{etoolbox}
\usepackage{nomencl}
\makenomenclature
\renewcommand\nomgroup[1]{%
  \item[\bfseries
  \ifstrequal{#1}{M}{Material Properties}{%
  \ifstrequal{#1}{P}{Physics Constants}{%
  \ifstrequal{#1}{O}{Symbols}{%
  \ifstrequal{#1}{Q}{Machine parameters}{%
  \ifstrequal{#1}{W}{Moulding parameters}{%
  \ifstrequal{#1}{A}{Abbreviations}{}}}}}}%
]}





